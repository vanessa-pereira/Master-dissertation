\documentclass[mim_thesis.tex]{subfiles}
\begin{document}
Nowadays, it is almost impossible for the population to image themselves without any technology to help on their daily basic routines or even to take care of their health. Since an early age, I had a huge interest in the electronic mechanisms in the medical area, mainly because these were so important to support my own life when I was a newborn. With that, I decided to do a licentiate degree in \ac{ECIM} at \ac{ISEP}. The interest in  information technology associated to the medical field started to appear from the previous course and in  2017 I decided to start my master studies in the \ac{MIM} at \ac{FMUP} and \ac{FCUP}. Also, I had the opportunity to work as a junior student researcher in the \ac{CINTESIS} and at Healthy Systems, a spin-off company from University of Porto focused on cybersecurity and performance consulting in healthcare systems. From there onward, I developed an interest in the standards of medical data and interoperability between different systems, mostly due to the opportunity of checking how it works in the real-world during this professional experience.\\ 

As an user with slightly preference on free open-source systems, I always had the idea that information and technology should be free and available to everyone. Once, I had the chance to listen a presentation about openEHR and the curiosity on this topic increased even more. So I planned to do a future study in this area and learn more about it. After some research on the actual market, I decided that the most advantageous option was to do again a mobility programme, now during the master degree under \textit{Erasmus + internship}. After applying, there was the possibility of making the dissertation at Marand d.o.o., a healthcare IT company located in Ljubljana, Slovenia, that mostly makes use of OpenEHR standard on their electronic health record systems (EHR). Along the academic year of 2017/2018, I have been working as an functional analyst and consultant for one of the projects inside of Marand, the Advanced Congestive Heart Failure (ACHF) where I have been doing several tasks, such as functional analysis, openEHR modeler and repository content owner, process and terminologies definer and functional tester for the web portal and mobile application of the project.\par

\begin{flushright}
\vfill {"Every path is the right path. \\ Everything could've been anything else.\\ And it would have just as much meaning.”}
\\― \textbf{from Mr. Nobody (2009)}
\end{flushright}
\end{document}