\documentclass[mim_thesis.tex]{subfiles} 
\begin{document}
In the future, it is expected to have a continuous improvement of this script, by adding new features to it, that can simplify even more the process of updating archetypes. Fortunately, the web script is having acceptance among the openEHR community and being used to update the artifacts for various repository projects. \\

Sometimes the changes on the archetypes can be only a new language added to the archetype and, in some of these cases, it is not really necessary to update the archetype locally. Usually a change in the major version of the archetype means the previous version was broken and should not be used anymore, and that's the main reason why the script is only, for now, searching and making comparison for major version. One of the future improvements is the implementation of the archetype parsing functionality, in order to get fully internal versioning differences from the parameters \textit{MD5-CAM}, \textit{build\_uid} and \textit{revision} in the archetypes of either repositories. These resources can also suffer smaller changes during a period of time, and these changes can be noticed by a difference on the previous parameters. Other functionality to be implemented is having a fully functional differences (\textit{"diff"}) comparison, that will allow to the side by side the different versions of archetypes from both repositories and check all the changes made on both. \\

Also, another improvement will be implementing a new module to the script to verify the commits made to the GitHub CKM, and in case of the message that follow that script is "The archetype id was changed.", make a list of what was the previous \textit{archetype\_id} and the new one, returning the links to download the new version.

\end{document}