\documentclass[mim_thesis.tex]{subfiles} 
\begin{document}

The aim of this dissertation was to study and find a suitable way of having the different openEHR based local repositories updated and compliant with the international openEHR CKM. To achieve it, the following studies were made:

\begin{enumerate}[noitemsep]
\item How the CKM works, historical information, features and functionalities and how the verification of new versions is made;
\item Study state of art methodologies used for managing software and document life-cycles;
\item Apply study to the proposed methodology for the current problem;
\item Development and implementation of the proposed methodology and verification of results. 
\end{enumerate}

Although the governance of the international openEHR CKM is managed on a continuous improvement way by an online community of mostly health care and informatics professionals, reviewing constantly the different artifacts, the same does not happen with the local repositories on several companies. When new EHR project based on openEHR standards have concluded the specifications and requirements phase and it is already prepared to get the development phase for the artifacts, the gathered information needs to be stored and always updated when necessary. The openEHR resources are not statical, a lot of new updates are made constantly to improve the knowledge saved and retrieved from them. With this, the expected outcomes were:

\begin{enumerate}[noitemsep]
\item Documentation of how to deal with the versioning of openEHR archetypes and templates on new and ongoing projects of local repositories;
\item Creation of a script that will automatically run on the local repositories connected to \textit{ADL Designer} to search new major versions of archetypes on CKM and allowing to download them to the local repository.
\end{enumerate}

It is expected that the methods that were implemented for updating these openEHR artifacts on local repositories, have a prevalence in the improvement of added value in the final application, which is dependent from the provided input data of the local repositories. 

\end{document}