\documentclass[mim_thesis.tex]{subfiles} 
\begin{document}

The aim of this dissertation is to study and find a suitable way of having the different openEHR based local repositories updated and compliant with the international openEHR CKM. To reach it, the following studies were made:

\begin{enumerate}[noitemsep]
\item How the CKM works, historical information, features and functionalities and how the verification of new versions is made;
\item Study state of art methodologies used for managing software and document life-cycles;
\item Apply study to the proposed methodology for current problem;
\item Development and implementation of the proposed methodology and verification of results. 
\end{enumerate}

Although the governance of the international openEHR CKM is managed on a continuous improvement way by an online community of mostly health care and informatics professionals, reviewing constantly the different artifacts, the same does not happen with the local repositories on several companies. It is expected that the methods that will be implemented for updating the openEHR artifacts on local SVN’s, where the artifacts are saved, have a prevalence in the improvement of added value in the final application, which is dependent from the provided input data of the local repositories. The expected outcomes are:

\begin{enumerate}[noitemsep]
\item Documentation of how to deal with the versioning of openEHR archetypes and templates on new and ongoing projects of local repositories;
\item Creation of a script that will automatically run on the local repositories to search new versions of archetypes on CKM and download them to the local repository, advising the user that the update of some archetype can make a change on different templates and show which of them will vary.
\end{enumerate}

 It is expected to contribute with an added value to the governance of openEHR resources.

\end{document}