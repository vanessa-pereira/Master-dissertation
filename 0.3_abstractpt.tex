\documentclass[mim_thesis.tex]{subfiles} 
\begin{document}

O openEHR é uma norma de especificação aberta e livre que providencia como guardar, partilhar e retornar dados de saúde, com a principal ideia de separar estes dados da implementação lógica das aplicações, de uma forma agnóstica. Possui o seu próprio repositório de governância de recursos ou artefactos (arquétipos e templates) chamado Clinical Knowledge Manager (CKM), que está sob constante revisão por uma comunidade online de profissionais em cuidados de saúde e informática. Apesar disso, existe um problema emergente. Cada empresa ou instituição que usa os arquétipos e templates do openEHR, não podem usar alguns destes artefactos directamente nos seus projetos individuais, já que as especificações de implementação exigem, com muita frequência, um grande nível de detalhe, muito localizado, que não faz parte dos conceitos comuns acordados internacionalmente.

Além disso, os implementadores de sistemas destas instituições que trabalham com recursos do openEHR para seus projetos, precisam ter uma versão funcional de todos os artefatos usados, ou seja, necessitam de ter o seu próprio repositório local com um sistema de arquivos ou um CVS (Control Version System) para gerenciá-los .

Estes repositórios são criados no início do projecto, através do download de cópias dos arquétipos disponíveis na CKM num determinado período de tempo e usualmente não são mais actualizados. Durante os anos, a CKM principal (também conhecida como openEHR CKM Internacional) teve nova versões actualizadas para diferentes arquétipos, tendo também uma gestão bem estruturada dos ciclos de vida dos arquétipos, com as novas versões a serem aprovadas depois do consenso de membros da comunidade \citep{Leslie2017}. Isto implica que o conhecimento da openEHR CKM está sempre actualizado, mas isso não acontece necessariamente nos repositórios locais.

Até agora, não havia uma definição clara de como resolver este problema: como gerir um repositório local, mantê-lo conectado com a openEHR CKM e manter o repositório local actualizado e em conformidade com este. Algumas empresas desenvolveram aplicativos e ferramentas de modelagem para trabalhar com os arquétipos, como o Archetype Editor da Ocean Informatics e, posteriormente, um aplicativo baseado em Web, o ADL Designer da Marand, que pode fazer a conexão web com diferentes repositórios baseados em artefatos openEHR e analisá-los.

Baseado neste problema, o objectivo do estudo foi encontrar e implementar uma forma adequada de ter diferente repositórios locais baseados em artefactos do openEHR actualizados e em conformidade com o repositório do openEHR CKM internacional. De modo a alcançar este objectivo, vários estudos foram feitos, tais como:
\begin{enumerate} [noitemsep]
\item Estudar como a openEHR CKM funciona, historia, características, funcionalidades e como a verificação de novas versões é feita;
\item Estudar o estado da arte de metodologias usadas para gerir software e documentação de ciclos de vida;
\item Rever as especificações do openEHR para o Modelo de Referência (RM) e Modelo de Arquétipos (AM), relacionado com a identificação e versionamento de arquétipos;
\item Criação de documentação sobre como lidar com o versionamento de arquétipos num repositório local e de um “script” para comparar os arquétipos contidos em ambos os repositórios e verificar as novas versões para cada arquétipo da openEHR CKM no GitHub.
\item Desenvolvimento e implementação da metodologia proposta e verificação de resultados.
\end{enumerate}

Inicialmente foi feita uma análise através da comparação entre arquétipos guardados num repositório local de um projecto em desenvolvimento e da versão actual do mesmo arquétipo na openEHR internacional, a qual consistiu numa comparação directa entre a versão “major” (V.x, onde x é um número inteiro correspondente à versão), parâmetros de versionamento e o conteúdo dos arquétipos. Deste repositório, \( \frac{1}{4} \) do número total de arquétipos (n=41) estão desactualizados e ter um repositório com estas caracteristicas não é o objectivo dos ideais da norma openEHR. Este repositório local para testes foi fornecido previamente e o conteúdo da openEHR CKM é replicado para um repositório de uma conta pública hospedada no GitHub, denominada “OpenEHR CKM international Mirror”. As ferramentas usadas foram compostas por: ADL Designer criada pela Marand, REST API do GitHub e do GitLab, openEHR CKM and Tortoise SVN. O script criado para a comparação de ambos os repositórios foi baseado na framework do Angular 2 e usada a linguagem TypeScript. Foram também usadas chamadas e métodos HTTP para as REST API do ADL Designer e GitHub.

A junção de toda a informação recolhida neste estudo permitiu criar uma metodologia e script para comparação de versões, juntamente com documentação sobre como lidar com o versionamento dos arquétipos do openEHR em projectos novos ou em desenvolvimento em repositórios locais. É expectável que contribua com um valor acrescido para a governância de recursos do openEHR.
Este estudo foi feito juntamente com a Faculdade de Medicina da Universidade do Porto (FMUP) em Portugal e a Marand d.o.o. na Eslovénia no âmbito do projeto em mobilidade “Erasmus + Estágios” de setembro de 2017 a setembro de 2018.\\ 

\textbf{Palavras-chave:} (MeSH) Electronic Health Records, Medical Informatics, Health Information Exchange, Health Information Management, (Não-MeSH) OpenEHR, Version Control



\end{document}