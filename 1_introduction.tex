\documentclass[mim_thesis.tex]{subfiles} 
\begin{document}
In a digital era, healthcare information had to adapt itself to the current technologies. The term “e-health” is used as a natural mode nowadays, which did not happen decades ago. \citep{Eysenbach2001} Gunther Eysenbach defined in 2001 this term as: \textit{“An emerging field in the intersection of medical informatics, public health and business, referring to health services and information delivered or enhanced through the Internet and related technologies. In a broader sense, the term characterizes not only a technical development, but also a state-of-mind, a way of thinking, an attitude, and a commitment for networked, global thinking, to improve healthcare locally, regionally, and worldwide by using information and communication technology”}. Also it was defined by him that "e" in e-health represents not only electronics, but also a set of words that are necessary to define these terms, such as efficiency, enhancing quality, evidence-based, empowerment, encouragement, education, enabling, extending, ethical, equity, easy-to-use, entertaining or even exciting.

Within this, an electronic health record can be considered as a subset of e-health. Since the 70’s, the usage of health records has been evolving from paper sheets to digital records. Although it improved a lot the medical access to patients data and corrected half of the errors on the same input data, the way of inserting information on an EHR can differ and this can result in some misunderstanding by the various users \citep{detmer1997computer}. Numerous new applications were developed and improved the way physicians interact with electronic health records \citep{Grandia2017}, but the underlying information model of how the patient data should be recorded and stored was left to be developed exclusively by implementers. If using a computer or a mobile application (tablets and phones) the modeling way of saving the patient data can differ in both ways and also can vary from software provider to another software provider, which installs their software in the different health providers. One of the difficulties in handling with the different systems is that most part of them do not use medical information standards. The software provider creates his own data models of saving data and expects this data to be exchanged with other systems from a different provider in other hospital, which mostly does not happen successfully or even does not happen at all. 

To address this issue, some standards emerged and from those, one stood up the most - the OpenEHR \citep{openEHRdeploy2017}. OpenEHR has its own international repository of clinical archetypes and templates called "Clinical Knowledge Manager" (CKM). However, a new issue has risen. Each company or institution working with openEHR archetypes and templates have their own local repository with a file system or a control version system (CVS) to manage them - usually by using \textit{GIT} or \textit{Subversion} systems. These local repositories are usually created at the start of a project by downloading copies of the available archetypes at a certain point in time and usually these are not updated anymore. Over the years, the main CKM (OpenEHR International CKM) had new updated versions for the different archetypes, having a very well structured management of archetypes lifecycles, with the new versions being approved after a community consensus. This implies that knowledge in CKM stays always updated, but that not does not necessarily happen on the local repositories. 

\end{document}