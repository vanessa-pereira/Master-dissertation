\documentclass[mim_thesis.tex]{subfiles} 
\begin{document}
Sample Chapter


ideias: 
sobre openehr ckm - site precisa de ser mais apelativo para uso, é complicado de usar para novos users, a revisao de arquetipos muitas vezes deixa passar varias lacunas e faz na mesma o update do arquetipo contendo erros
openehr ckm e github - ha arquetipos que nao foram enviados do openehr para o github (CONFIRMAR OUTRA VEZ) 


- sobre o script - ajuda a ser ser menos time consuming na verificacao dos arquetipos, usualmente as pessoas fazem download uma vez e deixam andar ate haver real necessidade de actualizar o arquetipo - tipo um bug na aplicacao que esta a usar este arquetipo, o que pode ser muito mau no caso de uma aplicacao ja em producao. o botao pode incentivar ao clique para verificacao rapida de novas versoes - a verificacao e comparacao manual de arquetipo por arquetipo num repositorio de 41 arquetipos com o repositorio da openehr ckm demorou cerca de 5h. 



\subsection{Major issues}

https://github.com/openEHR/CKM-mirror/tree/master/local/archetypes/cluster
openEHR-EHR-CLUSTER.address\_isa.v1 - existe na CKM e nao existe no github mirror 

ver todos os probs documentados no anexo

\end{document}